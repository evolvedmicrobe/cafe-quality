%%%%%%%%%%%%%%%%%%%%%%%%%%%%%%%%%%%%%%%%%
% Stylish Article
% LaTeX Template
% Version 2.0 (13/4/14)
%
% This template has been downloaded from:
% http://www.LaTeXTemplates.com
%
% Original author:
% Mathias Legrand (legrand.mathias@gmail.com)
%
% License:
% CC BY-NC-SA 3.0 (http://creativecommons.org/licenses/by-nc-sa/3.0/)
%
%%%%%%%%%%%%%%%%%%%%%%%%%%%%%%%%%%%%%%%%%

%----------------------------------------------------------------------------------------
%	PACKAGES AND OTHER DOCUMENT CONFIGURATIONS
%----------------------------------------------------------------------------------------

\documentclass[fleqn,10pt]{SelfArx} % Document font size and equations flushed left


%----------------------------------------------------------------------------------------
%	Source Code Listings
%----------------------------------------------------------------------------------------
\usepackage{listings}
\usepackage{xcolor}
\definecolor{darkgreen}{rgb}{0.0, 0.2, 0.13}
\definecolor{ao}{rgb}{0.0, 0.5, 0.0}
\lstdefinestyle{sharpc}{language=[Sharp]C, frame=single}
\lstset{% general command to set parameter(s)
basicstyle=\small, % print whole listing small
keywordstyle=\color{blue}\bfseries,
% underlined bold black keywords
commentstyle=\small\color{ao}, % white comments
stringstyle=\ttfamily, % typewriter type for strings
columns=fullflexible, % keeps the comments from exploding
showstringspaces=false} % no special string spaces


%----------------------------------------------------------------------------------------
%	COLUMNS
%----------------------------------------------------------------------------------------

\setlength{\columnsep}{0.55cm} % Distance between the two columns of text
\setlength{\fboxrule}{0.75pt} % Width of the border around the abstract

%----------------------------------------------------------------------------------------
%	COLORS
%----------------------------------------------------------------------------------------

\definecolor{color1}{RGB}{0,0,90} % Color of the article title and sections
\definecolor{color2}{RGB}{0,20,20} % Color of the boxes behind the abstract and headings

%----------------------------------------------------------------------------------------
%	HYPERLINKS
%----------------------------------------------------------------------------------------

\usepackage{hyperref} % Required for hyperlinks
\hypersetup{hidelinks,colorlinks,breaklinks=true,urlcolor=color2,citecolor=color1,linkcolor=color1,bookmarksopen=false,pdftitle={Title},pdfauthor={Author}}

%----------------------------------------------------------------------------------------
%	ARTICLE INFORMATION
%----------------------------------------------------------------------------------------

\JournalInfo{PBEP \#2, 2015} % Journal information
\Archive{PBEP Series} % Additional notes (e.g. copyright, DOI, review/research article)

\PaperTitle{PacBio Enhancement Proposal \#2 \\{\large Changing how merge QVs are used in secondary analysis}} % Article title

\Authors{Nigel Delaney} % Authors

\Keywords{} % Keywords - if you don't want any simply remove all the text between the curly brackets
\newcommand{\keywordname}{Keywords} % Defines the keywords heading name

%----------------------------------------------------------------------------------------
%	ABSTRACT
%----------------------------------------------------------------------------------------

\Abstract{ Merge QVs need work. }

%----------------------------------------------------------------------------------------

\begin{document}

\flushbottom % Makes all text pages the same height

\maketitle % Print the title and abstract box

\tableofcontents % Print the contents section

\thispagestyle{empty} % Removes page numbering from the first page

%----------------------------------------------------------------------------------------
%	ARTICLE CONTENTS
%----------------------------------------------------------------------------------------

\section*{Introduction} % The \section*{} command stops section numbering

\addcontentsline{toc}{section}{Introduction} % Adds this section to the table of contents


\subsection{What are Merge QVs?}
The MergeQV value is created within the method that converts pulses into bases\footnote{Method: \texttt{ AnalyzeInsertClassify@PulseToBaseStream.cs:671}}.

\lstset{style=sharpc}
\begin{lstlisting}[frame=single]
// The mergeQV is provided by the pulse caller
mergeQV.Add(pulseMergeQV[i]);
\end{lstlisting}

Tracing further up it appears this value is set in \texttt{TraceToPulse:454} by this code:

\lstset{style=sharpc}
\begin{lstlisting}[frame=single]
// The mergeQV is provided by the pulse caller
// Pulse-call quality values
mergeQvList.Add((byte) Math.Round(
	Math.Min(Math.Max(pulse.MergeQV, 0.0f), 255f)) );

\end{lstlisting}

This seems to be mainly set in the FrameCrfClassiier\footnote{\texttt{FrameCrfClassifier.cs:600}}

\lstset{style=sharpc}
\begin{lstlisting}[frame=single, float=*]
/// Estimate the quality score for a given segment's classification
private float ComputeQualityValue(int label, int rhsLabel, int startFrame, int endFrame, out float mergeQv)
{
  // Probability of a miscall. If label=0, P(deletion).  
  // Otherwise, includes P(insertion), based on comparison to the P(label=0) case.
  var miscallProb = MiscallProb(label, rhsLabel, startFrame, endFrame);  

  // Probability that the called pulse includes a merge error (label != 0 case);
  // or, Probability that a No-Pulse region missed a short (single-frame) pulse.
  var mergeOrMissProb = MergeOrMissProb(label, rhsLabel, startFrame, endFrame);  

  if (label == 0)
  {
      mergeQv = LogPeToPhredQv(new LFloat(float.Epsilon)); 
      return LogPeToPhredQv(miscallProb + mergeOrMissProb);
  }  

  // Otherwise, for a pulse label, return separate label and merge results
  mergeQv = LogPeToPhredQv(mergeOrMissProb);
  return LogPeToPhredQv(miscallProb);
}       
\end{lstlisting}

%------------------------------------------------

\section{Principled Arguments For MergeQV Improvements}



\subsection{They make scoring complex}

The ``Merge Move" in our template scoring routine is done in one of two distinct ways depending on if a DelTag is present.  With positions denoted by Figure \ref{fig:del}, a deletion move is currently scored according to the equation shown below.  

\begin{figure}[ht]\centering % Using \begin{figure*} makes the figure take up the entire width of the page
\includegraphics[width=\linewidth]{Merge}
\caption{Merge Scoring}
\label{fig:del}
\end{figure}

\[
 \alpha_{i-1,j-2}  +  \begin{cases}
			\text{Merge score from } R_{i}  & \text{if }  R_{i} = T_{j} \text{ \& } R_{i} = T_{j-1} \\
			\text{A constant (typically }-\infty\text{)} & \text{otherwise}
			\end{cases}
\]


%------------------------------------------------

\section{Empirical Arguments Against DelTags}


%------------------------------------------------


\section{Conclusions and Possibilities}



%----------------------------------------------------------------------------------------

\end{document}